%%% Здесь выбираются необходимые графы
\documentclass[russian,utf8,pointsection,nocolumnsxix,nocolumnxxxi,nocolumnxxxii]{eskdtext}

%%% Для работы со сложными формулами
\usepackage{amsmath}
\usepackage{amssymb}

%%% Что бы использовать символ градуса (°) - \celsius
\usepackage{gensymb}

%%% Что бы работал eskdx и некоторые другие пакеты LaTeX
\usepackage{xecyr}

%%% Для работы шрифтов
\usepackage{xltxtra}
%%% Ставим Times New Roman - как основной шрифт
\setmainfont[Mapping=tex-text]{Times New Roman}
%%% Courier New - для моноширного текста
\setmonofont[Scale=MatchLowercase]{Courier New}

%%% Для того чтобы работали стандартные сочетания символов ---, --, << >> и т.п.
\defaultfontfeatures{Mapping=tex-text}

%%% Для работы с русскими текстами (расстановки переносов)
\usepackage{polyglossia}
\setdefaultlanguage{russian}
\newfontfamily\russianfont{Times New Roman}

%%% Для переноса составных слов
\XeTeXinterchartokenstate=1
\XeTeXcharclass `\- 24
\XeTeXinterchartoks 24 0 ={\hskip\z@skip}
\XeTeXinterchartoks 0 24 ={\nobreak}

%%% Ставим подпись к рисункам. Вместо «рис. 1» будет «Рисунок 1»
\addto{\captionsrussian}{\renewcommand{\figurename}{Рисунок}}
%%% Убираем точки после цифр в заголовках
\def\russian@capsformat{%
  \def\postchapter{\@aftersepkern}%
  \def\postsection{\@aftersepkern}%
  \def\postsubsection{\@aftersepkern}%
  \def\postsubsubsection{\@aftersepkern}%
  \def\postparagraph{\@aftersepkern}%
  \def\postsubparagraph{\@aftersepkern}%
}



% Автоматически переносить на след. строку слова которые не убираются
% в строке
\sloppy

%%% Для вставки рисунков
\usepackage{graphicx}

%%% Для вставки интернет ссылок, полезно в библиографии
\usepackage{url}
